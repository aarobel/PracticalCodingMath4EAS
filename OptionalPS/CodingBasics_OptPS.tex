\documentclass{article}
\usepackage[T1]{fontenc}
\usepackage{amsmath,amssymb,epsfig,wrapfig,xfrac}
\usepackage{natbib}
\usepackage{lineno,chapterbib}
%\usepackage[pdftex]{graphicx}
\graphicspath{{./}}
\oddsidemargin=0.1in
\bibliographystyle{apalike}
\usepackage[pdftex,bookmarks,colorlinks=true,plainpages=false, 
citecolor=black,urlcolor=blue,filecolor=blue]{hyperref} % usage: \href{URL}{text}
\newcommand*\samethanks[1][\value{footnote}]{\footnotemark[#1]}
\newcommand{\mytilde}{\raise.17ex\hbox{$\scriptstyle\mathtt{\sim}$}}
\newcommand{\pd}[2]{\frac{\partial {#1}}{\partial {#2}}}
\newcommand{\od}[2]{\frac{\mathrm{d} {#1}}{\mathrm{d} {#2}}}
\newcommand{\pdl}[2]{{\partial {#1}}/{\partial {#2}}}
\newcommand{\beq}{\begin{equation*}}
\newcommand{\eeq}{\end{equation*}}
\newcommand{\be}{\begin{enumerate}} 
\newcommand{\ee}{\end{enumerate}}

\setlength{\topmargin}{-0.5in}

\setlength{\textwidth}{6.5in}
\setlength{\textheight}{9.1in}
\title{\vspace{-1in} Optional PS: Coding Basics}
\begin{document}

\maketitle

\be
 \item The sum $\mathcal{S}_k$ is given by
\beq
\mathcal{S}_k= 
\sum_{n=0}^k \frac{(-1)^n}{(2n+1)^7}\qquad .
\eeq

\be
\item Write a function that calculates the value of $\mathcal{S}_k$ (as a function of k). Output $S_k$ for $k=10$, $k=100$, $k=1000$ in a for loop. 
\item Write a new function that calculates the value of $\mathcal{S}_k$ (as a function of k) with only vector operations (i.e. vectorize the code from part a). Output $S_k$ for $k=10$, $k=100$, $k=1000$. 
\item Add a capability to your script to write both, the index $k$ and the value of $S_k$ in a comma-separated file \emph{drl.csv}.
\item One can show that
\beq
\mathcal{S}_\infty= 
\sum_{n=0}^\infty\frac{(-1)^n}{(2n+1)^7}=\frac{61\pi^7}{184320}\qquad .
\eeq
Write a script with a while loop to determine the \emph{smallest} value of $k$ which fulfills 
\beq
\left|\mathcal{S}_k-\mathcal{S}_\infty \right|< 10^{-4}\qquad.
\eeq
\ee
\item Write a script that solves the following linear system of equations using matrix operations:
\begin{eqnarray*}
2x+y-4z & = & -5 \\
3x-y+9z &=& 5\\
5x+2y + 2z &=& -1\quad .
\end{eqnarray*}
Give the solution $(x,y,z)$.

\item Consider the 3-D surface
\beq
z(x,y) = \sin \left(\pi x + 2 \pi y \right) + e^{-5 y^2}
\eeq

\be
\item Create a mesh for $x$ and $y$ over the intervals $x = [0,1]$ and  $y = [0,3]$ with 101 grid points in $x$ and 301 grid points in $y$.
\item Create a new array calculating values of $z$ on this mesh.
\item Find every mesh point where $0.5<z(x,y)<1$ within the domain $x = [0,1]$ and $y = [0,1]$ on the mesh created in part (a).
\item Calculate the maximum value of a new function $w(x,y)$ within the part of the mesh where 0.5<z(x,y)<1, where
\beq
w(x,y) = \cos \left(\pi x + \pi y \right) + e^{-2 y^2}
\eeq

\ee

\ee 
\end{document}
\end{enumerate}
\end{document}