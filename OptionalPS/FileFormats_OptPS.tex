\documentclass{article}
\usepackage[T1]{fontenc}
\usepackage{amsmath,amssymb,epsfig,wrapfig,xfrac}
\usepackage{natbib}
\usepackage{lineno,chapterbib}
%\usepackage[pdftex]{graphicx}
\graphicspath{{./}}
\oddsidemargin=0.1in
\bibliographystyle{apalike}
\usepackage[pdftex,bookmarks,colorlinks=true,plainpages=false, 
citecolor=black,urlcolor=blue,filecolor=blue]{hyperref} % usage: \href{URL}{text}
\newcommand*\samethanks[1][\value{footnote}]{\footnotemark[#1]}
\newcommand{\mytilde}{\raise.17ex\hbox{$\scriptstyle\mathtt{\sim}$}}
\newcommand{\pd}[2]{\frac{\partial {#1}}{\partial {#2}}}
\newcommand{\od}[2]{\frac{\mathrm{d} {#1}}{\mathrm{d} {#2}}}
\newcommand{\pdl}[2]{{\partial {#1}}/{\partial {#2}}}
\newcommand{\beq}{\begin{equation*}}
\newcommand{\eeq}{\end{equation*}}
\newcommand{\be}{\begin{enumerate}} 
\newcommand{\ee}{\end{enumerate}}

\setlength{\topmargin}{-0.5in}

\setlength{\textwidth}{6.5in}
\setlength{\textheight}{9.1in}
\title{\vspace{-1in} Optional PS: File Formats}
\begin{document}

\maketitle

Download \texttt{hurricane\_data\_after2000.csv} from class GitHub into the same folder where you plan to write code.

\be
\item Read the hurricane data into a Pandas dataframe.

\item Make ISO\_TIME the index on your dataframe

\item Find the unique values of the BASIN, SUBBASIN, and NATURE columns.

\item Rename the WMO\_WIND and WMO\_PRES columns to WIND and PRES.

\item Get the 10 largest rows in the dataset by WIND.

\item Group the data on SID and get the 10 largest hurricanes by WIND.

\item Make a bar chart of the wind speed of the 20 strongest-wind hurricanes.

\item Plot the count of all datapoints by Basin

\item Plot the count of unique hurricanes by Basin.

\item Make a \href{https://pandas.pydata.org/docs/reference/api/pandas.DataFrame.plot.hexbin.html}{hexbin} of the location of datapoints in Latitude and Longitude.

\item Plot the count of all datapoints per year as a timeseries using resample

\item Calculate the climatology of datapoint counts as a function of dayofyear

\ee
\end{document}
\end{enumerate}
\end{document}