\documentclass{article}
\usepackage[T1]{fontenc}
\usepackage{amsmath,amssymb,epsfig,wrapfig,xfrac}
\usepackage{natbib}
\usepackage{lineno,chapterbib}
%\usepackage[pdftex]{graphicx}
\graphicspath{{./}}
\oddsidemargin=0.1in
\bibliographystyle{apalike}
\usepackage[pdftex,bookmarks,colorlinks=true,plainpages=false, 
citecolor=black,urlcolor=blue,filecolor=blue]{hyperref} % usage: \href{URL}{text}
\newcommand*\samethanks[1][\value{footnote}]{\footnotemark[#1]}
\newcommand{\mytilde}{\raise.17ex\hbox{$\scriptstyle\mathtt{\sim}$}}
\newcommand{\pd}[2]{\frac{\partial {#1}}{\partial {#2}}}
\newcommand{\od}[2]{\frac{\mathrm{d} {#1}}{\mathrm{d} {#2}}}
\newcommand{\pdl}[2]{{\partial {#1}}/{\partial {#2}}}
\newcommand{\beq}{\begin{equation*}}
\newcommand{\eeq}{\end{equation*}}
\newcommand{\be}{\begin{enumerate}} 
\newcommand{\ee}{\end{enumerate}}

\setlength{\topmargin}{-0.5in}

\setlength{\textwidth}{6.5in}
\setlength{\textheight}{9.1in}
\title{\vspace{-1in} PS: Interpolation, Smoothing and Fourier Series}
\date{}
\begin{document}

\maketitle

\noindent Download hourly sea level data from Charleston, SC using \href{https://tidesandcurrents.noaa.gov/waterlevels.html?id=8665530}{the NOAA tide \& currents website}. Select a time range of the last year, ``1 hr'' interval and have it output ``Data Only''. Then click ``export to CSV''.

\be
\item Read in the sea level data, creating separate vectors for the date and the verified sea level.

\item Calculate the Fourier coefficients of this data and note the frequencies of the dominant tides.

\item Smooth the data with a boxcar window of 24 hours and plot the resulting smoothed time series over a time series of the original data, zooming in to a one week period of data.

\item Smooth the data with Gaussian windows of 12 and 24 hours and plot the resulting smoothed time series on top of the previous plot.

\item Calculate the daily average tidal cycle

\item Calculate and plot 1 month of anomalies from the mean daily tidal cycle

\item Filter out all periodicities less than 1 day and the plot the result one year filtered time series

\ee

\noindent Now using the monthly sea level data from Charleston from the last two problem sets, calculate a smoothed time series, filtering out sub-annual variability in a way that you choose. Use various statistics on the filtered data to determine if the long-term rate of sea level rise has changed, and if so, estimate when it changed.

\end{document}
\end{enumerate}
\end{document}