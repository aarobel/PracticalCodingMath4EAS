\documentclass{article}
\usepackage[T1]{fontenc}
\usepackage{amsmath,amssymb,epsfig,wrapfig,xfrac}
\usepackage{natbib}
\usepackage{lineno,chapterbib}
%\usepackage[pdftex]{graphicx}
\graphicspath{{./}}
\oddsidemargin=0.1in
\bibliographystyle{apalike}
\usepackage[pdftex,bookmarks,colorlinks=true,plainpages=false, 
citecolor=black,urlcolor=blue,filecolor=blue]{hyperref} % usage: \href{URL}{text}
\newcommand*\samethanks[1][\value{footnote}]{\footnotemark[#1]}
\newcommand{\mytilde}{\raise.17ex\hbox{$\scriptstyle\mathtt{\sim}$}}
\newcommand{\pd}[2]{\frac{\partial {#1}}{\partial {#2}}}
\newcommand{\od}[2]{\frac{\mathrm{d} {#1}}{\mathrm{d} {#2}}}
\newcommand{\pdl}[2]{{\partial {#1}}/{\partial {#2}}}
\newcommand{\beq}{\begin{equation*}}
\newcommand{\eeq}{\end{equation*}}
\newcommand{\be}{\begin{enumerate}} 
\newcommand{\ee}{\end{enumerate}}

\setlength{\topmargin}{-0.5in}

\setlength{\textwidth}{6.5in}
\setlength{\textheight}{9.1in}
\title{\vspace{-1in} PS: Statistics}
\date{}
\begin{document}

\maketitle

Download monthly sea level data from Charleston, SC using \href{https://psmsl.org/data/obtaining/rlr.monthly.data/234.rlrdata}{this PSMSL dataset}. You can download it directly or copy and paste it into a .txt file.

\be
\item Read in the sea level data, creating separate vectors for the date (first column) and the sea level (second column).

\item Plot the histograms of sea level data for these time periods on top of each other: 1923-1942, 2003-2022

\item Calculate mean May sea level for each of these two periods

\item Calculate t-values for each of these time periods

\item Evaluate whether we can reject the null hypothesis at a 95\% confidence level that the mean May sea level from each of these periods is the same as the long-term mean of May sea level (and if so, at what level) 


\ee
\end{document}
\end{enumerate}
\end{document}