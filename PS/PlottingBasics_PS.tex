\documentclass{article}
\usepackage[T1]{fontenc}
\usepackage{amsmath,amssymb,epsfig,wrapfig,xfrac}
\usepackage{natbib}
\usepackage{lineno,chapterbib}
%\usepackage[pdftex]{graphicx}
\graphicspath{{./}}
\oddsidemargin=0.1in
\bibliographystyle{apalike}
\usepackage[pdftex,bookmarks,colorlinks=true,plainpages=false, 
citecolor=black,urlcolor=blue,filecolor=blue]{hyperref} % usage: \href{URL}{text}
\newcommand*\samethanks[1][\value{footnote}]{\footnotemark[#1]}
\newcommand{\mytilde}{\raise.17ex\hbox{$\scriptstyle\mathtt{\sim}$}}
\newcommand{\pd}[2]{\frac{\partial {#1}}{\partial {#2}}}
\newcommand{\od}[2]{\frac{\mathrm{d} {#1}}{\mathrm{d} {#2}}}
\newcommand{\pdl}[2]{{\partial {#1}}/{\partial {#2}}}
\newcommand{\beq}{\begin{equation*}}
\newcommand{\eeq}{\end{equation*}}
\newcommand{\be}{\begin{enumerate}} 
\newcommand{\ee}{\end{enumerate}}

\setlength{\topmargin}{-0.5in}

\setlength{\textwidth}{6.5in}
\setlength{\textheight}{9.1in}
\title{\vspace{-1in} PS: Plotting Basics}
\date{}
\begin{document}

\maketitle

Download \texttt{ATL\_MonMeanTemp\_1879\_2020.csv} and \texttt{SEA\_MonMeanTemp\_1894\_2020.csv} files from class GitHub into the same folder where you plan to write code.

\be
\item Read in the Atlanta and Seattle temperature data into separate matrices/arrays. In MATLAB, use \texttt{csvread} or the ``Import Data'' utility. In Python use \texttt{numpy.genfromxt}.

\item Plot the time series of Atlanta September temperature data from 1900 to 2020. Make sure to label all axes and make everything readable.

\item Plot the time series of Seattle September temperature data from 1900 to 2020, making sure first to replace any missing data with a NaN. Make sure to label all axes and make everything readable. 

\item Plot all Atlanta data as a contour or pcolor plot with year on x-axis and month on y-axis. Use a sensible colormap and label axes.

\item Plot all Seattle data as a contour or pcolor plot with year on x-axis and month on y-axis. Use a sensible colormap and label axes. Be careful with missing data.

\item In a row of two subplots (one for each city), make box plots showing the range of monthly temperature. Make sure the y-axes are the same so the difference between the cities is obvious.

\item Plot a histogram of July temperatures in Atlanta, then plot a curve on top of it using the following equation for a Gaussian:
\begin{equation}
\frac{1}{\sqrt{2 \pi} \sigma} \exp \left[-\frac{1}{2} \left(\frac{x-\mu}{\sigma} \right)^2 \right]
\end{equation}
where $\sigma$ and $\mu$ are the standard deviation and mean calculated from the data with appropriate functions.

\item Make a scatter plot of Atlanta temperatures vs Seattle temperatures for corresponding months.

\ee
\end{document}
\end{enumerate}
\end{document}