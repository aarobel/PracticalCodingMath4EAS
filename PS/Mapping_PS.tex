\documentclass{article}
\usepackage[T1]{fontenc}
\usepackage{amsmath,amssymb,epsfig,wrapfig,xfrac}
\usepackage{natbib}
\usepackage{lineno,chapterbib}
%\usepackage[pdftex]{graphicx}
\graphicspath{{./}}
\oddsidemargin=0.1in
\bibliographystyle{apalike}
\usepackage[pdftex,bookmarks,colorlinks=true,plainpages=false, 
citecolor=black,urlcolor=blue,filecolor=blue]{hyperref} % usage: \href{URL}{text}
\newcommand*\samethanks[1][\value{footnote}]{\footnotemark[#1]}
\newcommand{\mytilde}{\raise.17ex\hbox{$\scriptstyle\mathtt{\sim}$}}
\newcommand{\pd}[2]{\frac{\partial {#1}}{\partial {#2}}}
\newcommand{\od}[2]{\frac{\mathrm{d} {#1}}{\mathrm{d} {#2}}}
\newcommand{\pdl}[2]{{\partial {#1}}/{\partial {#2}}}
\newcommand{\beq}{\begin{equation*}}
\newcommand{\eeq}{\end{equation*}}
\newcommand{\be}{\begin{enumerate}} 
\newcommand{\ee}{\end{enumerate}}

\setlength{\topmargin}{-0.5in}

\setlength{\textwidth}{6.5in}
\setlength{\textheight}{9.1in}
\title{\vspace{-1in} PS: Mapping}
\date{}
\begin{document}

\maketitle
\be 
\item Download \texttt{sst.ltm.nc} from class GitHub into the same folder where you plan to write code. This is the long-term mean of SST from observations interpolated onto a rectangular grid.

\be
\item Load the SST data into your platform of choice (MATLAB or Python/Xarray)

\item Initialize a map using a Mercator projection with continents filled in with grey shading. Add rivers as thick blue lines.

\item Plot the mean SST data as a filled contour plot using a sequential colormap.

\item Plot the SST data as anomalies from the global mean as a filled contour plot using a divergent colormap.

\item Make another map of SST anomalies from the global mean using a more sensible projection (something that doesn't exaggerate the poles)

\item Make a map of regional SST within 500 km of Savannah, GA using a projection that is appropriate for the smaller domain. Plot elevation over land areas, rivers and indicate the location of Savannah with a marker and a text label.
\ee

\item Download \texttt{ARGOtraj.zip} from class GitHub into the same folder where you plan to write code. This includes a series of NetCDF files with the ARGO drifter buoy location data near the East Coast of the US for the first few weeks of August 2023.

\be
\item Load all the drifter latitude/longitude data into a reasonable data type in your preferred platform.

\item Make a regional plot of the Western Atlantic using a reasonable projection. Shade the coastline.

\item Plot all drifter data on this map with each drifter plotted as a different color.

\item Make another plot with all drifter data with each trajectory colored by the speed of the drifter at that time. Make sure to include a colorbar with a reasonable colormap.

\ee

\ee
\end{document}
\end{enumerate}
\end{document}