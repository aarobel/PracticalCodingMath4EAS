\documentclass{article}
\usepackage[T1]{fontenc}
\usepackage{amsmath,amssymb,epsfig,wrapfig,xfrac}
\usepackage{natbib}
\usepackage{lineno,chapterbib}
%\usepackage[pdftex]{graphicx}
\graphicspath{{./}}
\oddsidemargin=0.1in
\bibliographystyle{apalike}
\usepackage[pdftex,bookmarks,colorlinks=true,plainpages=false, 
citecolor=black,urlcolor=blue,filecolor=blue]{hyperref} % usage: \href{URL}{text}
\newcommand*\samethanks[1][\value{footnote}]{\footnotemark[#1]}
\newcommand{\mytilde}{\raise.17ex\hbox{$\scriptstyle\mathtt{\sim}$}}
\newcommand{\pd}[2]{\frac{\partial {#1}}{\partial {#2}}}
\newcommand{\od}[2]{\frac{\mathrm{d} {#1}}{\mathrm{d} {#2}}}
\newcommand{\pdl}[2]{{\partial {#1}}/{\partial {#2}}}
\newcommand{\beq}{\begin{equation*}}
\newcommand{\eeq}{\end{equation*}}
\newcommand{\be}{\begin{enumerate}} 
\newcommand{\ee}{\end{enumerate}}

\setlength{\topmargin}{-0.5in}

\setlength{\textwidth}{6.5in}
\setlength{\textheight}{9.1in}
\title{\vspace{-1in} PS: Ordinary Differential Equations}
\date{}
\begin{document}

\maketitle
\be 
\item Consider the ordinary differential equation
\begin{equation}
\frac{dx}{dt} = rx - x^3
\end{equation}

\be
\item Use a numerical ODE solver to solve this problem with $r = 4$ in the above equation and an initial condition $x(t=0) = 1$ over the time interval $t = [0, 10]$. Plot the numerical solution $x(t)$ over this time interval.

\item Use the same method from above, but with the initial condition $x(t=0) = 3$. Plot the numerical solution. 

\item Use the same method from above, but with the initial condition $x(t=0) = -1$. Plot the numerical solution. 

\item Now use a numerical ODE solver to solve the problem with $r = -4$ in the above equation and an initial condition $x(t=0) = 1$ and also $x(t=0) = -1$  over the time interval $t = [0, 10]$. Plot the two numerical solutions $x(t)$ over this time interval on the same plot.

\item Discuss the different solutions that occur for different initial conditions, and what happen when $r$ changes from positive to negative.

\item Use a symbolic solver to solve for the exact solution to the ODE above, and plot the solution on a plot with a numerical solution for $r = 4$ and $x(t=0) = 1$.

\ee



\ee
\end{document}
\end{enumerate}
\end{document}